\chapter{Overview}
\label{opening_remarks}
The core of the Earth begins at around $2890$ km beneath the surface and extends to the center of the planet. The outer two-thirds (extending from $2890$ km down to about $5150$ km) makes up the liquid outer core, which is composed primarily of molten iron \citep{lW20}. 
The liquid core is constantly in motion, and its dynamics are important for a number of geophysical processes. 
Convection in the mantle, which ultimately drives plate tectonics, is partially controlled by the heat flux out of the outer core \citep{gS01}. 
The Earth's length of day is known to fluctuate on decadal and sub-decadal timescales, and interactions between the solid earth and liquid core are known to contribute \citep{sP02,sZ97}.
Perhaps the greatest consequence of the outer core is the geodynamo, the global scale magnetic field that protects Earth from hazardous solar radiation.
It is believed that electrical currents within the core generate the geodynamo, and
a large body of analytical (eg. \citep{eP55,gR72}), observational (eg. \citep{cV84,aJ00}), and numerical (eg. \citep{gG95a,gG95b,uC06}) work supports this idea to the extent that it is almost universally accepted\citep{pR13}. 

Calculations suggest that viscous and Ohmic dissipation would arrest flows within the outer core in less than $10^{6}$ years \citep{aC98}. Therefore, any consistent description of core fluid processes requires a persistent forcing mechanism to sustain the motions and magnetic field.
A likely candidate is the crystallization of the solid inner core, which makes up the innermost structure of the Earth. The inner core is slowly cooling. It releases this heat into the surrounding outer core. As the temperature of the inner core drops, the interface between the inner and outer core freezes, releasing latent heat and light elements into the outer core. 
Warm, light fluid elements at the inner core boundary are lifted towards the cold mantle, while cold heavy fluid sinks towards the inner core. 
This process facilitates fluid motion, mixing, and the generation of the geodynamo and is generally referred to as natural convection. 

Natural convection is also relevant to terrestrial atmospheres and oceans \citep{gV06}-- where it plays a role in climate and weather-- and is ubiquitous within and beyond the Solar System. The interiors of stars, many planets, exo-planets, and satellites \citep{yA21,tG25,jK22,dL23,dN25} are all expected to contain convecting fluids.
As such, the physical problem has received significant attention, beginning with the work of Boussinesq, Oberbeck, B{\'e}nard, and Rayleigh around the turn of the 20th century \citep{fB89}.
Unfortunately, the processes involved are remarkably complex, and much is still unknown about natural convection.
The fluid processes are chaotic, indicating that the prediction of exact outcomes, such as the amount of precipitation from an imminent storm, is impossible.
Before the advent of high powered computing, it was exceedingly difficult to model the chaotic processes which we believe govern many geophysical processes.
Even now it remains challenge, andmodern efforts utilize state of the art computing resources and often require hundreds to thousands of processors. 
Much has been learned, but we are still unable to reach the parameter regimes we expect to be relevant to a planet.
Rather than attempt to provide exact predictions for a particular event or system, the most fruitful approach is often to predict statistics and emergent phenomena. 
%In preparation for a storm, for example, we are able to forecast rainfall and wind speeds, but not when and where the largest gust will occur.

In this dissertation I present a collection of studies regarding turbulent natural convection. 
The primary physical system of interest is the liquid outer core of the Earth, although many of the results are applicable to astrophysical, terrestrial, and even industrial processes.
In Chapter \ref{Introduction} I will introduce the fluid mechanics of geophysical systems, and provide necessary background on turbulent convection and the mathematical systems used to model it.
Chapter \ref{Oliver_2023} discusses a numerical study of an idealized fluid-system undergoing rapid rotation. 
Our findings suggest that previously reported results regarding the balance of forces within the fluid system are incomplete, and we provide new theory which highlights the importance of viscosity in rapidly-rotating convective systems.
Chapters \ref{Oliver_2025} and \ref{Oliver_2026} investigate the role of topographic coupling between the mantle and the outer core. 
We consider local and global topographic shapes and discuss the impact of topography on flow morphology and topographic torques. 
We provide scaling results which suggest that topographic torques are sufficiently large to explain variations in the Earth's length of day.
