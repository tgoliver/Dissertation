\documentclass{article}
\usepackage{amsmath}
\usepackage{graphicx}
\usepackage{float}
\usepackage{hyperref}
\usepackage{xcolor}
\usepackage{subfig}
\usepackage{biblatex}

\def\epo{\epsilon\rightarrow 0}
\def\lb{\left(}
\def\rb{\right)}
\def\ls{\left[ \vphantom]}
\def\rs{\right] }
\def\ep{\epsilon}

\title{}
\author{}
\date{}

\begin{document}
\section{The science we do between}
Over the course of these last five years
%during which I have pursued this degree
it has been my distinct pleasure to be surrounded by a group of dear colleagues who, at the risk of incriminating them or myself, I count amongst my great friends. Like me, they are all conspirators in a quest for doctoral gratification, and each has demonstrated themselves to be a scientist of the most capital offence.

Science is broad and complex, and despite countless office chats, lab visits, coffee breaks, tea times, and lunches, I could likely only provide the most basic explanation for the research that any of them do. I hope that their own efforts towards my work would provide similarly fruitless, lest I should suffer the great shame of publishing understandable science.
Fortunately, it is not the details of our work in which we find common ground. 
Rather, we share an appreciation for the strange, the silly, and the bizarre that Nature, in her immaculate quest to fulfill the laws of physics, relentlessly forces upon us.

And this common ground can be found anywhere.
Certainly it can be found in the experimentalist's lab or the theorist's office, but just as well can the absurd be caught lurking in the living room, or hanging from a chair lift, or sitting in some sticky corner of the Dark Horse bar. For example, the claim ``One can soft boil an egg in the microwave'' is a hypothesis best suited for the kitchen and someone else's microwave.

Perhaps these investigations are fringe or foolish, perhaps they are a waste of time, or just what we do when we are not working, but surely we have learned from them, surely we have satiated our curiosities, and surely it has been fun. It is the science we do between, and I have done it with my fine friends.
Therefore, I would like to take a moment to recognize these collaborators, and mention some of their remarkable findings over these last five years.

To Mr. Zack Sierzega, who I first \textcolor{red}{met over a bottle of bourbon during a TA training hour}, thank you for your pioneering experiments on broken symmetries in cross-country skiing. You were the first to investigate hybrid ski systems. No one had ever dared to test predictions that coupling classic and skate skis might reduce efficiency. It was an uphill challenge that you conquered one leg at a time; first a push, then a glide.

To Mr. John Wilson, who \textcolor{red}{planned to be here an hour ago and is therefore busy with something else,} thank you for your relentless trials in the field of culinary science. I will admit initial skepticism about your techniques, but there is no question that you have convinced me that the scattering properties of boiled chicken are indeed interesting. And that study produced results for many years. Every once in a while, under the couch, or tucked behind the TV, a new finding would emerge.

To Ms. Lane Terry, who \textcolor{red}{was willing to put up with Mr. Wilson and Mr. Sierzega when I no longer could,} thank you for your contributions to veterinary and social sciences. The sacrifice of your many living spaces for groundbreaking publications such as ``Perspectives on Doug" and ``Can a horse live indoors?" will forever be appreciated by the community.

To Mr. Pablo Aramburu Sanchez, who \textcolor{red}{delighted us all with his pursuit of romance,} thank you for your enlightening theories in sports science and diversity studies. He was the first to propose an analytical phase transition in the watchability of women's basketball. Despite the many objections of his colleagues and a harsh peer review, he persevered. 

To Mr. Jesse Krusse \textcolor{red}{Maybe something about ALTA}?

And finally, to Mr. River Leversee, who \textcolor{red}{destroyed his car on my darkest day of grad school just to make me feel better}, I can never hope to compile a complete list of the many unhinged achievements of your illustrious career. Nevertheless, a few deserve special mention. For confectionery arts, thank you for your generous donation of sugar equivalent to the GDP of a small island nation. 
For civil engineering, you performed a novel stress test of the United States Postal Service and it's capacity to handle a condensed system of plastic vinyl Anatidae. Your results are immortalized in a brief correspondence with collaborators: ''I had a ton of [rubber] ducks delivered." 
And finally for your foundational biomedical study ``Can one man induce diabetes by consuming 10\% of his body weight in white rice in a single sitting?" We may never know...

Thank you all for your many lessons. Please never let science become too serious. 
\end{document}
