\chapter{Introduction}
\label{Introduction}
\section{Earth's Core and other Planetary Interiors}
\label{s:pi}
\subsection*{The core}
\label{s:thecore}
The core of the Earth is comprised of two, distinct layers. 
The inner core is solid and has a radius near $1230$ km \citep{eE74}.
The liquid outer core extends from the inner-core boundary to the rocky mantle, located about $2900$ km beneath the Earth's surface. 
Combined, the two layers make up $32\%$ of the mass of the Earth, but only $16\%$ of the volume. 
This discrepancy is because the core is predominantly composed of heavy iron (Fe) and, to a lesser extent, nickel (Ni) \citep{hK13}.
A host of lighter elements--sulfur (S), carbon (C), silicon (Si), oxygen (O), and hydrogen (H)--likely make up the remaining $\sim 10\%$ by mass \citep{fB52,jP94}. 

The inner core is solid due to the tremendous pressures at the deepest layers of the planet. Little is known about the inner-core's structure, however estimates suggest that it first nucleated somewhere between 0.5 and 1.5 billion years ago \citep{sL15,cD15,aB15}.

The outer core is liquid. At depths shallower than the inner core boundary (ICB) pressures are sufficiently decreased such that iron exists in the liquid phase. 
Experiments on the melting point of iron at relevant pressures suggest that the temperature of the ICB is around $5400$ K, although significant uncertainty exists\citep{rB93,aL00,jN04,dA09}.
As the planet cools, the inner core freezes, releasing heat and light elements into the outer core. 
This mechanism drives turbulent convection in the liquid layer, and powers the geodynamo.
In this work, the coupling between inner and outer core will always be relegated to boundary conditions on the outer core fluid. 
This approach significantly simplifies calculations and captures the key physical process, that is the transport of thermal and chemical buoyancy into the fluid region.

The core mantle boundary (CMB) is the exterior terminus of the core. 
Located at a depth of around 2900 km, the CMB is the interface between the silicate mantle and the core and is significantly colder than the ICB, around $4200$ K\citep{lS09,gF10}.

The composition and structure of the CMB has attracted significant attention, largely because the CMB drives convection in the mantle, which is ultimately responsible for plate tectonics\citep{gS01}.
However these considerations also bear significance on the core side. 
A convectively stable layer at the CMB, for example, may help to stabilize the geomagnetic field \citep{bS08}. 
Chapters \ref{Oliver_2025} and \ref{Oliver_2026} of this dissertation are primarily concerned with the effects of topography at the CMB as it relates to core fluid motions.
As is the case with the ICB, all of the work presented here will relegate CMB interactions to the role of boundary conditions on the fluid.

Along with the solid Earth, the core rotates on a 24-hour period. 
It is largely accepted that the inner core co-rotates as well, which is facilitated via ICB coupling. Some studies (eg. \citep{jV25}) suggest differential rotation, although the effect is a few degrees per year-- significant, but far less than the daily rotation.
Core rotation plays a major role in determining the dynamics of the core.
Due to the large length scales and small fluid viscosities relevant to geophysical bodies, conservation of angular momentum constrains fluid motions to be largely aligned with the rotation axis of the planet \citep{sC61}.
The usual mathematical prescription is to work in a frame of reference co-rotating with the planet. 
This approach yields a host of so-called ``fictitious forces,'' such as centrifugal, Coriolis, and Eulerian forces, which arise in the treatment of any rotating coordinate system. 
The Coriolis effect certainly plays the largest role of the three and is responsible for the axial constraints on the flow.
All studies presented here consider a rapidly rotating fluid such that the Coriolis acceleration is large and primarily balanced by pressure. 
This condition is referred to as geostrophy and is relevant to the core (eg. \citep{kZ07}) and many other planetary applications including terrestrial oceans and atmospheres \citep{gV06}.

The primarily metallic core is also electrically conductive.
It therefore hosts electric currents which generate large magnetic fields \citep{aC98, sC61}. 
This process is responsible for the geodynamo, a largely dipolar magnetic field roughly aligned with the Earth's rotation axis (the rotation axis and dipole axis differ by about $11^{\circ}$). 
Geophysical systems are dissipative, which means that they slowly lose electrical and mechanical energy to heat. 
It is believed that the Earth has had an active dynamo for at least 2.8 billion years \citep{jA07},
however electrical (Ohmic) dissipation is though to be strong enough to stifle an un-powered dynamo within 10 million years \citep{aC98}. 
Mechanical forcing (convection in the case of the Earth) is necessary to provide the power source that sustains the magnetic field.
The dynamo's behavior is therefore tied to turbulent fluid processes. Indeed it exhibits significant and unpredictable dynamics. 
Excursions (in which the field loses dipolar power only to regain it) and reversals (in which the field's orientation flips) are well documented in the geologic record. An average rate of reversals is estimated to be around three each million years \citep{jJ94c1}, although the behaviour is seemingly stochastic, making prediction of future reversals difficult. 

The studies presented in this dissertation are hydrodynamic only, and therefore do not include magnetic effects.  
Nevertheless, the results can still be applied to the dynamo problem, as the magnetic and hydrodynamic problems share many similarities.  

\subsection*{Beyond the core}
Although special to us, there does not seem to be anything particularly unique about the Earth's liquid core. 
Liquid and fluid layers are ubiquitous within and beyond the solar system.
Convection plays a critical role in the formation and evolution of stars. In the Sun, the convection zone, which begins at a depth of around $200,000$ km and extends to the surface, powers the solar dynamo and transports heat and light to the exterior.
The Solar cycle, a near 11 year oscillation in the number of sunspots, is controlled by the convection zone (eg. \citep{eP55,mO03}).
 
The gas giants convect, and many of the Jovian and Saturnian moons are expected to host dynamic, subsurface oceans. 
Jupiter in particular provides a tremendous laboratory for rotating convection, as the fluid is visible at the surface.
Data from the Voyager missions determined that Jupiter's bands are counter-rotating jets powered by the interior (eg.\citep{pG86,sL86}). 
Recent results from the Juno mission have identified that these jets are aligned with the rotation axis, similar to predictions for the Earth's core, and tabletop weather experiments\citep{eG21}.
Europa, Enceladus, Ganymede, and other icy moons likely contain salty oceans which has implications for the possibility of extra-terrestrial life \citep{fN16}.

\section{Modeling geophysical fluid systems}
\label{s:gafd}
\begin{center}
\textit{``...whatever we do affects everything and everyone else, if even in the tiniest way. Why, when a housefly flaps his wings, a breeze goes round the world..."}\\
\end{center}
--The Princess of Pure Reason, The Phantom Tollbooth
\footnote{Published 1961, coincidentally the same year Ed Lorenz mistakenly demonstrated chaos on a Royal McBee LGP-30} 

As is the standard procedure in fluid mechanics, our approach begins with the governing equations for the quantities of interest--in this case mass, energy, and momentum. A complete description of the formulation of the equations of fluid mechanics can be found in most introductory texts (eg. \citep{dT01,gV06}), although some comments on the momentum equation should be made before progressing to a more specific discussion on convection.
In a frame undergoing constant rotation about the origin, the momentum equation is
\begin{equation}
	\frac{\partial \mathbf{u}}{\partial t} + \mathbf{u}\cdot \nabla \mathbf{u} = -\frac{1}{\rho}\nabla p^{*}- 2 \mathbf{\Omega}\times\mathbf{u} + \nu\nabla^{2}\mathbf{u}+ \frac{\rho^{*}}{\rho}\mathbf{g},
	\label{e:ns_f}
\end{equation}
which is a particular expression of the Navier-Stokes equation.
The fluid velocity is the vector $\mathbf{u}$, t the time,  $\rho$ the fluid density, $p^{*}$ the pressure, $\mathbf\Omega$ the rotation vector,  $\mathbf{r}$ the position vector from the origin, $\nu$ the fluid kinematic viscosity, and $\mathbf{g}$ the gravitational acceleration vector (which may be a function of $\mathbf{r}$). 
$\nabla$ and $\nabla^{2}$ are the gradient and laplacian operators respectively. The left-hand side (LHS) of eqn. (\ref{e:ns_f}) is often abbreviated
$\frac{D\mathbf{u}}{Dt},$ 
where the operator  $\frac{D}{Dt} = \frac{\partial }{\partial t} + \mathbf{u} \cdot \nabla$ is known as the material derivative.
Throughout this dissertation we will use the common notation $\mathbf{u} = \ls u,v,w\rs$ to refer to the $\hat{\mathbf{x}},$ $\hat{\mathbf{y}},\text{ and }\hat{\mathbf{z}}$ components of the flow. Note that the oceanography and atmospheric communities often use $w$ to refer to the planet surface-normal velocity. When considering spherical geometries we will use $u_{r}$ to refer to this flow.

The second term on the LHS is non-linear, which presents the foremost complication involved with solving eqn. (\ref{e:ns_f}). 
Briefly put, the non-linear term permits interactions between large and small structures within the flow such that any attempt to solve must involve a procedure for handling a broad range of length scales. 
In the context of planetary interiors, this would mean resolving scales that span the planet all the way down to molecular distances where viscous effects arrest the flow. 
It is the basis of the ``Butterfly effect,'' which suggests that the flap of a butterfly's wings may determine the path of a hurricane weeks later.

Obviously such complexities are not tractable, and thus some system is necessary for resolving this difficulty. 
Indeed the entire scope of this dissertation can be summarized as developing strategies to overcome just this obstacle. 

To start, we employ the Boussinesq approximation, which prescribes a constant density to all terms except gravity\citep{sC61}. Furthermore, we ignore centrifugal effects, which tend to be small in geophysical applications. 
This permits a simplified version of eqn. \eqref{e:ns_f}.
\begin{equation}
	\frac{\partial \mathbf{u}}{\partial t} + \mathbf{u}\cdot \nabla \mathbf{u} = -\nabla p- 2 \mathbf{\Omega}\times\mathbf{u} + \nu\nabla^{2}\mathbf{u}+ \rho^{*}\mathbf{g}
	\label{e:ns}
\end{equation}
$\rho^{*}$ is the density fluctuation normalized by the reference density. It has no units. We have absorbed $\rho$ into the pressure gradient so that $p$ is not technically a pressure. Nevertheless, we will refer to $p$ as the pressure since it is the dynamically relevant quantity for incompressible flows (note that it also accounts for the hydrostatic weight of the fluid)\citep{sC61}.
Solving eqn. \eqref{e:ns} for planetary parameters is still infeasible however.
Our approach will involve solving eqn. (\ref{e:ns}) under achievable circumstances and then developing scaling arguments which can be extended to the regimes of planets.

\subsection*{Geostrophy and Rotation} 
Planets, including Earth, tend to rotate very quickly such that viscous and buoyant effects cannot balance the Coriolis acceleration.
In this limit, eqn. \eqref{e:ns} satisfies
\begin{equation}
	2\mathbf{\Omega}\times \mathbf{u} \approx -\nabla p.
	\label{e:geos}
\end{equation}
The implication is that the streamlines of the horizontal flow are nearly coincident with the isobars. 
For sufficiently large $\mathbf{\Omega},$ fluctuating flows, even those that are turbulent, obey eqn. (\ref{e:geos}) at leading order. 
As long as eqn. (\ref{e:geos}) is satisfied, we refer to the flow as ``quasi-gesotrophy'' or, for large $\mathbf{u}$, ``geostrophic turbulence.''

Curling  eqn. \eqref{e:geos} and leveraging incompressibility yields the Taylor-Proudman constraint
\[2\lb\mathbf{\Omega}\cdot\nabla\rb \mathbf{u} \approx 0, \]
which states that geostrophic flows are invariant along the direction of rotation. 
For turbulent flows, Taylor-Proudman is a constraint, not a law, and flows are allowed to deviate. 
Nevertheless, geostrophic turbulence is always characterised by flows that are highly anisotropic. 
They can vary rapidly in directions perpendicular to rotation, but vary slowly in the rotation direction, and boundary effects are translated axially through the domain so that very few effects are truly localized.

An important consequence is that convection, which generally requires velocities along the rotation axis, is inhibited by rotation. 
To what extent remains an outstanding question, although much progress has been made particularly through analysis of the rotating Rayleigh-B\`enard system.


\subsection*{The rotating Rayleigh-B\`enard problem}
Consider a rotating fluid bounded between two infinite plates whose temperature's are fixed such that they have a temperature difference $\Delta T$. 
The plates are separated by a distance $H.$ 
Gravity and rotation are anti-parallel and oriented such that gravity points from cold to warm plate. The coordinate system is chosen such that $z = 0$ corresponds to the bottom plate and $\hat{\mathbf{z}}$ is parallel to $\mathbf{\Omega}.$ 
It is useful to non-dimensionalize the system\citep{dT01}. We scale lengths and temperatures by $H$ and $\Delta T$ respectively. To best illustrate the role of geostrophy, we chose a viscous time-scale $H^{2}/\nu$ and a velocity scale of $\nu/H.$
The non-dimensional equations are
\begin{subequations}
	\begin{equation}
		\frac{D}{Dt}\mathbf{u} = -\nabla P-\frac{2}{Ek}\hat{\mathbf{z}}\times \mathbf{u} + \nabla^{2}\mathbf{u} + \frac{Ra}{Pr}T\mathbf{\hat{z}},\\
	\label{e:rbc_1}
\end{equation}
\begin{equation}	
		\nabla\cdot \mathbf{u} = 0,\\
	\label{e:rbc_2}
\end{equation}
\begin{equation}	
	\frac{D}{Dt}T = \frac{1}{Pr} \nabla^{2}T.
	\label{e:rbc_3}
\end{equation}
	\label{e:rbc_eqn_nd}
\end{subequations}
The pressure $P$ has been re-defined to include the hydrostatic and centrifugal terms, the latter of which is assumed much smaller than the former \citep{sC61}.
The Ekman, Rayleigh, and Prandtl numbers are dimensionless parameters defined as
\[Ek = \frac{\nu}{\Omega H^{2}}\qquad Ra = \frac{g\alpha \Delta T H^{3}}{\nu\kappa} \qquad Pr =\frac{\nu}{\kappa}.\]
The Ekman number is a measure of the influence of rotation, with smaller values indicating larger Coriolis accelerations. 
The Rayleigh number indicates the strength of thermal forcing, and is usually quite large, often exceeding $10^{20}$ in planetary cores (eg. \citep{dL23}). The Prandtl number is a fluid property and is the ratio of momenta and thermal diffusivities. 
The Prandtl number of water is around $7,$ whereas values for the core are estimated to be between $10^{-1}-10^{-2}$ \citep{gW98}.

Eqns. (\ref{e:rbc_eqn_nd}) are accompanied by boundary conditions on the temperature and velocity. 
To avoid the complication of Ekman layers (see \citep{hG68})  
we will consider the ``stress-free'' condition on the velocity field for the present discussion, although Chapters \ref{Oliver_2025} and \ref{Oliver_2026} deal exclusively with ``no-slip'' boundaries. 
The boundary conditions are 
\begin{subequations}
\begin{equation}
T =1\text{ at } z=0 \qquad T =0 \text{ at } z=1
	\label{e:bc_t}
\end{equation}
\begin{equation}
	w = \frac{\partial u}{\partial z} = \frac{\partial v}{\partial z}=0 \quad\text{ at }\quad z = 0,1.
	\label{e:bc_vsf}
\end{equation}
\label{e:bc}	
\end{subequations}

It is also useful to define some \textit{a posteriori} quantities to diagnose flow speeds. They are the Reynolds number $Re$ and Rossby number $Ro.$ They quantify the magnitude of the velocity field to viscous and Coriolis effects respectively. Their definitions are
 \[Re = \frac{U H}{\nu}\qquad Ro=EkRe =\frac{U}{H\Omega},\] 
 where $U$ is a characteristic velocity.
 In the present non-dimensionalization, $Re = \left|\mathbf{u}\right|,$ where $\left|\cdot\right|$ is some kind of norm, most often an rms.



