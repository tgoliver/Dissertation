\chapter{Introduction}
\label{Introduction}
\section{Earth's Core and other Planetary Interiors}
\label{s:pi}
\subsection*{The core}
\label{s:thecore}
The core of the Earth is comprised of two, distinct layers. 
The inner core is solid and has a radius near $1230$ km \citep{eE74}.
The liquid outer core extends $2260$ km from the inner-core boundary to the rocky mantle, located about $2900$ km beneath the Earth's surface. 
Combined, the two layers make up $32\%$ of the mass of the Earth, but only $16\%$ of the volume. 
This discrepancy is because the core is predominantly composed of heavy iron (Fe) and, to a lesser extent, nickel (Ni) \citep{hK13}.
A host of lighter elements--sulfur (S), carbon (C), silicon (Si), oxygen (O), and hydrogen (H)--likely make up the remaining $10\%$ by mass \citep{fB52,jP94}. 

The inner core is solid due to the tremendous pressures at the deepest layers of the planet. Little is known about the inner-core's structure, however estimates suggest that it first nucleated somewhere between 0.5 and 1.5 billion years ago \citep{aB15,cD15,sL15}.

The outer core is liquid. At depths shallower than the inner core boundary (ICB) pressures are sufficiently reduced such that iron exists in the liquid phase. 
Experiments on the melting point of iron at relevant pressures suggest that the temperature of the ICB is around $5400$ K, although significant uncertainty exists \citep{dA09,rB93,aL00,jN04}.
As the planet cools, the inner core freezes, releasing heat and light elements into the outer core. 
This mechanism drives turbulent convection in the liquid layer, and powers the geodynamo.
In this work, the coupling between inner and outer core will always be relegated to boundary conditions on the outer core fluid. 
This approach significantly simplifies calculations and captures the key physical process, that is the transport of thermal and chemical buoyancy into the fluid region.

The core mantle boundary (CMB) is the exterior terminus of the core. 
Located at a depth of around 2900 km, the CMB is the interface between the silicate mantle and the core and is significantly colder than the ICB, around $4200$ K \citep{lS09,gF10}.

The composition and structure of the CMB has attracted significant attention, largely because the CMB drives convection in the mantle, which is ultimately responsible for plate tectonics \citep{gS01}.
However these considerations also bear significance on the core side. 
Chapters \ref{Oliver_2025} and \ref{Oliver_2026} of this dissertation are primarily concerned with the effects of topography at the CMB as it relates to core fluid motions.
As is the case with the ICB, all of the work presented here will relegate CMB interactions to the role of boundary conditions on the fluid.

The core rotates on a 24-hour period along with the rest of the planet. 
It is widely accepted that the inner core co-rotates, a process facilitated by ICB coupling. Some studies (eg. \citep{jV25}) suggest differential rotation, although the effect is a few degrees per year-- significant, but far less than the daily rotation.
Rotation plays a major role in the dynamics of the core.
Due to the large length scales and small fluid viscosities relevant to geophysical bodies, conservation of angular momentum constrains fluid motions to be largely aligned with the rotation axis of the planet \citep{sC61}.
The usual mathematical prescription is to work in a frame of reference co-rotating with the planet. 
This approach yields a host of so-called ``fictitious forces,'' such as centrifugal, Coriolis, and Eulerian forces, which arise in the treatment of any rotating coordinate system. 
The Coriolis effect certainly plays the largest role of the three and is responsible for axial constraints on the flow.
All studies presented here consider a rapidly rotating fluid such that the Coriolis force is large and primarily balanced by the pressure gradient. 
This condition is referred to as geostrophy and is relevant to the core (eg. \citep{kZ07}) and many other planetary applications including terrestrial oceans and atmospheres \citep{gV06}.

The core is made up of mostly metal and is therefore electrically conductive.
It hosts electric currents which generate large magnetic fields \citep{aC98, sC61}. 
This process is responsible for the geodynamo, a largely dipolar magnetic field roughly aligned with the Earth's rotation axis (the rotation axis and dipole axis differ by about $11^{\circ}$). 
Geophysical systems are dissipative, which means that they slowly lose electrical and mechanical energy to heat. 
It is believed that the Earth has had an active dynamo for at least 2.8 billion years \citep{jA07},
however electric (Ohmic) dissipation is though to be strong enough to stifle an un-powered dynamo within 10 million years \citep{aC98}. 
Mechanical forcing is necessary to provide the power source that sustains the magnetic field. Thermal or compositional convection are the most likely candidates \citep{sB95,dG77}, although tidal \citep{rK18} and precessional \citep{dC19} forcing have been investigated and likely contribute to the dynamics.
Regardless, the dynamo's behaviour is tied to turbulent fluid processes. Indeed it exhibits significant and unpredictable dynamics. 
Excursions (in which the field loses dipolar power only to regain it) and reversals (in which the field's orientation flips) are well documented in the geologic record \textcolor{red}{cit.}. An average rate of reversals is estimated to be around three each million years \citep{jJ94c1}, although the behaviour is seemingly stochastic, making prediction of future reversals difficult. 

The studies presented in this dissertation are hydrodynamic only, and therefore do not include magnetic effects.  
Nevertheless, the results can still be applied to the dynamo problem, as the magnetic and hydrodynamic problems share many similarities.  
\subsection*{Topography at the core mantle boundary}
\begin{figure}
	\begin{center}
		\includegraphics[width=0.5\textwidth]{Introduction/figures/llsvp_lekic}
	\end{center}
	\caption[Vote map of Large Low Shear Velocity Provinces]{Vote map for the presence of Large Low Shear Velocity Provinces (LLSVPs). Warmer colors correspond to greater consensus about the presence of LLSVPs. Borrowed from Lekic et al. (2012) \citep{vL12}}
	\label{f:llsvp_lekic}
\end{figure}
Chapters \ref{Oliver_2025} and \ref{Oliver_2026} are motivated by hypothesized topography at the core mantle boundary (CMB).
There is good reason to expect that the CMB is not a perfectly smooth surface. 
Like the surface of the Earth, we believe the CMB to be bumpy.
But CMB topography can be placed on firm geodynamical footing because a vast catalog of seismic evidence suggests the presence of multiple, large heterogeneities at the base of the mantle which may press and relieve the CMB.

Regions deep in the mantle, beneath southern Africa and the southern Pacific, are associated with slow seismic velocities (eg. \citep{aM19,jR99,jR04}). They are dubbed the Large Low Shear Velocity Provinces (LLSVPs) and are 1000s of kilometers in lateral extent and 100s in depth.
There is strong agreement within the community regarding the existence and locations of the LLSVPs.
Figure \ref{f:llsvp_lekic} displays the likely location of these structures.
Nevertheless their origin remains a mystery.
The main competing theories are that the LLSVPs are either compositionally (eg. \citep{yL14b,pT98}), or thermally (eg. \citep{dD12,gS04}) distinct from the surrounding mantle.
Regardless, their presence suggests density heterogeneities in the mantle.
Dynamical studies of mantle convection have demonstrated that the LLSVPs generate CMB topography \citep{tL07} with amplitudes ranging around $10$ km \citep{tL10}, although seismic evidence suggest smaller amplitudes of about $2$ km \citep{sT10}.

Chapters \ref{Oliver_2025} and \ref{Oliver_2026} discuss the ``topographic torques," in which core flow interacts with topography to exert torques on the mantle \citep{rH69,pR12}. 
Because CMB topography is not well constrained, we chose to study simplified topographies. Chapter \ref{Oliver_2025} considers a single bump, while Chapter \ref{Oliver_2026} considers single spherical harmonics. 
However, we also study a model informed by the LLSVPs. 
In this model, we place topography wherever LLSVPs are present (following \citep{vL12}). 
This approach is ad-hoc, and likely does not represent real topography, however no modeling or seismic approach has definitively mapped CMB topography. 
Therefore, we have simply chosen a model that is global, contains a broad spectrum, and is associated with the LLSVPs.

Furthermore, we study topographies with dimensional amplitudes between $10$ and $500$ km, much larger than the geophysical predictions. 
We do this because our simulations are much more viscous than the actual core, and we therefore must use topographies large enough such that they extend far beyond viscous boundary layers. 
In particular the Ekman layer, which is estimated to be $<O(1$m$)$ in the core (eg. \citep{hG68}), is at least  $1000$ times larger in our simulations. 
For this reason, we perform a suite of simulations at varied topographic amplitude and determine a scaling relationship which we extrapolate to the core.

\subsection*{Beyond the core}
Although special to us, there does not seem to be anything particularly unique about the Earth's liquid core. 
Liquid and fluid layers are ubiquitous within and beyond the solar system.
Convection plays a critical role in the formation and evolution of stars. In the Sun, the convection zone, which begins at a depth of around $200,000$ km and extends to the surface, powers the solar dynamo and transports heat and light to the exterior.
The Solar cycle, a near 11 year oscillation in the number of sunspots, is controlled by the convection zone (eg. \citep{eP55,mO03}).
 
The gas giants convect, and many of the Jovian and Saturnian moons are expected to host dynamic, subsurface oceans. 
Jupiter in particular provides a tremendous laboratory for rotating convection, as the fluid is visible at the surface.
Data from the Voyager missions determined that Jupiter's bands are counter-rotating jets powered by the interior (eg. \citep{pG86,sL86}). 
Recent results from the Juno mission have identified that these jets are aligned with the rotation axis, similar to predictions for the Earth's core, and tabletop weather experiments \citep{eG21}.
Europa, Enceladus, Ganymede, and other icy moons likely contain salty oceans which has implications for the possibility of extra-terrestrial life \citep{fN16}.

\section{Modeling geophysical fluid systems}
\label{s:gafd}
\begin{center}
\textit{``...whatever we do affects everything and everyone else, if even in the tiniest way. Why, when a housefly flaps his wings, a breeze goes round the world..."}\\
\end{center}
--The Princess of Pure Reason, The Phantom Tollbooth
\footnote{Published 1961, coincidentally the same year Ed Lorenz mistakenly demonstrated chaos on a Royal McBee LGP-30} 

As is the standard procedure in fluid mechanics, our approach begins with the governing equations for the quantities of interest--in this case mass, energy, and momentum. A complete description of the formulation of the equations of fluid mechanics can be found in most introductory texts (eg. \citep{dT01,gV06}), although some comments on the momentum equation should be made before progressing to a more specific discussion on convection.
In a frame undergoing constant rotation about the origin, the momentum equation is
\begin{equation}
	\frac{\partial \mathbf{u}}{\partial t} + \mathbf{u}\cdot \nabla \mathbf{u} = -\frac{1}{\rho}\nabla p^{*}- 2 \mathbf{\Omega}\times\mathbf{u}+ \mathbf{\Omega}\times\mathbf{\Omega}\times \mathbf{r} + \nu\nabla^{2}\mathbf{u}+ \frac{\rho^{*}}{\rho}\mathbf{g},
	\label{e:ns_f}
\end{equation}
which is a particular expression of the Navier-Stokes equation.
The fluid velocity is the vector $\mathbf{u}$, t the time,  $\rho$ the fluid density, $p^{*}$ the pressure, $\mathbf\Omega$ the rotation vector,  $\mathbf{r}$ the position vector from the origin, $\nu$ the fluid kinematic viscosity, and $\mathbf{g}$ the gravitational acceleration vector (which may be a function of $\mathbf{r}$). 
$\nabla$ and $\nabla^{2}$ are the gradient and laplacian operators respectively. The left-hand side (LHS) of eqn. (\ref{e:ns_f}) is often abbreviated
$\frac{D\mathbf{u}}{Dt},$ 
where the operator  $\frac{D}{Dt} = \frac{\partial }{\partial t} + \mathbf{u} \cdot \nabla$ is known as the material derivative.
Throughout this dissertation we will use the common notation $\mathbf{u} = u \hat{\mathbf{x}} + v \hat{\mathbf{y}} + w \hat{\mathbf{z}}$ to refer to the $\hat{\mathbf{x}},$ $\hat{\mathbf{y}},\text{ and }\hat{\mathbf{z}}$ components of the flow. 
We choose $\zhat$ to align with $\mathbf{\Omega},$ and at times it is useful to discuss the velocity field perpendicular to rotation.
We use the notation $\mathbf{u}_{\perp}= u \xhat + v \yhat$. 
Note that the oceanography and atmospheric communities often use $w$ to refer to the planet surface-normal velocity. When considering spherical geometries we will use $u_{r}$ to refer to this flow.

The second term on the LHS, known as the advective term, is non-linear, which presents the foremost complication involved with solving eqn. (\ref{e:ns_f}). 
Briefly put, the advection term permits interactions between large and small structures within the flow such that any attempt to solve must involve a procedure for handling a broad range of length scales. 
In the context of planetary interiors, this would mean resolving scales that span the planet, all the way down to molecular distances where viscous effects arrest the flow. 
It is the basis of the ``Butterfly effect,'' which suggests that the flap of a butterfly's wings may determine the path of a hurricane weeks later.
Obviously such complexities are not tractable, and thus some system is necessary for resolving this difficulty. 
%Indeed the entire scope of this dissertation can be summarized as developing strategies to get around this obstacle. 

To start, we employ the Boussinesq approximation, which prescribes a constant density to all terms except gravity \citep{sC61}. Furthermore, we ignore centrifugal effects, which tend to be small in geophysical applications. 
This permits a simplified version of eqn. \eqref{e:ns_f}.
\begin{equation}
	\frac{\partial \mathbf{u}}{\partial t} + \mathbf{u}\cdot \nabla \mathbf{u} = -\nabla p- 2 \mathbf{\Omega}\times\mathbf{u} + \nu\nabla^{2}\mathbf{u}+ \rho^{*}\mathbf{g}
	\label{e:ns}
\end{equation}
$\rho^{*}$ is the density fluctuation normalized by the reference density. It has no units. We have absorbed $\rho$ into the pressure gradient so that $p$ is not technically a pressure. Nevertheless, we will refer to $p$ as the pressure since it is the dynamically relevant quantity for incompressible flows (note that it also accounts for the hydrostatic weight of the fluid) \citep{sC61}.
The advective term, however, is unchanged, so solving eqn. \eqref{e:ns} for planetary parameters remains infeasible.
Our approach will involve solving eqn. (\ref{e:ns}) under achievable circumstances and then developing scaling arguments which can be extended to the regimes of planets.

\subsection*{Geostrophy and Rotation} 
Planets, including Earth, tend to rotate very quickly such that viscous and buoyant effects cannot balance the Coriolis acceleration.
In this limit, eqn. \eqref{e:ns} satisfies
\begin{equation}
	2\mathbf{\Omega}\times \mathbf{u} \approx -\nabla p.
	\label{e:geos}
\end{equation}
The implication is that the streamlines of the horizontal flow are nearly coincident with the isobars. 
For sufficiently large $\mathbf{\Omega},$ fluctuating flows, even those that are turbulent, obey eqn. (\ref{e:geos}) at leading order. 
As long as eqn. (\ref{e:geos}) is satisfied, we refer to the flow as ``quasi-gesotrophy'' or, for large $\mathbf{u}$, ``geostrophic turbulence'' \citep{jC71}.
We distinguish between regimes by calcuating two \textit{a posteriori} quantities to diagnose flow speeds. They are the Reynolds number $Re$ and Rossby number $Ro.$ 
They quantify the magnitude of inertia $\lb \mathbf{u}\cdot\nabla\mathbf{u}\rb $ to viscous and Coriolis accelerations respectively. Their definitions are
 \[Re = \frac{U H}{\nu}\qquad Ro =\frac{U}{H\Omega},\] 
 where $U$ is a characteristic velocity, most commonly an rms, and $H$ a characteristic length scale.
 The geostrophic regime is characterised by $Ro\ll1.$ In addition to small $Ro,$ \textit{geostrophic turbulence} requires $Re\gg1$.

Curling  eqn. \eqref{e:geos} and leveraging incompressibility yields the Taylor-Proudman constraint
\begin{equation}
2\lb\mathbf{\Omega}\cdot\nabla\rb \mathbf{u} \approx 0
\label{e:tp}
\end{equation}

which states that geostrophic flows are invariant along the direction of rotation. 
For turbulent flows, Taylor-Proudman is a constraint, not a law, and flows are allowed to deviate. 
Nevertheless, geostrophic turbulence is always characterised by flows that are highly anisotropic. 
They can vary rapidly in directions perpendicular to rotation, but vary slowly in the rotation direction.
Boundary effects are translated axially through the domain so that very few flow structures are truly localized.

An important consequence is that convection, which generally requires velocities along the rotation axis, is inhibited by rotation. 
To what extent remains an outstanding question, although much progress has been made particularly through analysis of the rotating Rayleigh-B\`enard system.


\subsection*{The rotating Rayleigh-B\`enard problem}
In this section we discuss a canonical model for rotating convection.
Consider a rotating fluid bounded between two infinite plates whose temperature's are fixed such that they have a temperature difference $\Delta T$. 
The plates are separated by a distance $H.$ 
Gravity and rotation are anti-parallel and oriented such that gravity points from cold to warm plate. The coordinate system is chosen such that $z = 0$ corresponds to the bottom plate and $\hat{\mathbf{z}}$ is parallel to $\mathbf{\Omega}.$ 
It is useful to non-dimensionalize the system \citep{dT01}. We scale lengths and temperatures by $H$ and $\Delta T$ respectively. To best illustrate the role of geostrophy, we choose a viscous time-scale $H^{2}/\nu$ and a velocity scale of $\nu/H.$
The non-dimensional equations are
\begin{subequations}
	\begin{equation}
		\frac{D}{Dt}\mathbf{u} = -\nabla p-\frac{2}{Ek}\hat{\mathbf{z}}\times \mathbf{u} + \nabla^{2}\mathbf{u} + \frac{Ra}{Pr}T\mathbf{\hat{z}},\\
	\label{e:rbc_1}
\end{equation}
\begin{equation}	
		\nabla\cdot \mathbf{u} = 0,\\
	\label{e:rbc_2}
\end{equation}
\begin{equation}	
	\frac{D}{Dt}T = \frac{1}{Pr} \nabla^{2}T.
	\label{e:rbc_3}
\end{equation}
	\label{e:rbc_eqn_nd}
\end{subequations}
All variables are dimensionless. This will be the convention for the remainder of this thesis unless otherwise noted. 
The Ekman ($Ek$), Rayleigh $\lb Ra\rb $, and Prandtl $\lb Pr\rb $ numbers are dimensionless parameters defined as
\[Ek = \frac{\nu}{\Omega H^{2}}\qquad Ra = \frac{g\alpha \Delta T H^{3}}{\nu\kappa} \qquad Pr =\frac{\nu}{\kappa}.\]
The Ekman number is a measure of the influence of rotation, with smaller values indicating larger Coriolis accelerations. 
The Rayleigh number indicates the strength of thermal forcing, and is usually quite large, often exceeding $10^{20}$ in planetary cores (eg. \citep{dL23}). The Prandtl number is a fluid property and is the ratio of kinematic viscosity to thermal diffusivity. 
The Prandtl number of water is around $7,$ whereas values for the core are estimated to be between $10^{-1}-10^{-2}$ \citep{gW98}.
In the present non-dimensionalization, $Re \sim \left|\mathbf{u}\right|$. The definition of $Ek$ allows us to calculate the Rossby number as $Ro = EkRe.$ 

Table \ref{t:nondim_geo} provides estimate values for the Earth, as well as a few modern studies considered state of the art. In the Appendix, Table \ref{at:param} provides a comprehensive list of definitions and values for dimensional and non-dimensional quantities within the core. 
The primary limitations in numerical studies for reaching Earth-like parameters are resolution constraints.
Experiments tend to be limited by the size of the apparatus and the volume of working fluid. 
Neither are capable of reaching the $Ek$ and $Ra$ regimes of the core.


Eqns. (\ref{e:rbc_eqn_nd}) are accompanied by boundary conditions on the temperature and velocity. 
To avoid the complication of Ekman layers (see \citep{hG68})  
we will consider the ``stress-free'' condition on the velocity field for the present discussion.
The stress-free boundary conditions are 
\begin{subequations}
\begin{equation}
T =1\text{ at } z=0 \qquad T =0 \text{ at } z=1
	\label{e:bc_t}
\end{equation}
\begin{equation}
	w = \frac{\partial u}{\partial z} = \frac{\partial v}{\partial z}=0 \quad\text{ at }\quad z = 0,1.
	\label{e:bc_vsf}
\end{equation}
\label{e:bc}	
\end{subequations} 
Chapters \ref{Oliver_2025} and \ref{Oliver_2026} deal exclusively with ``no-slip'' boundaries. Although not relevant to this discussion, they are included below for completeness.
\begin{equation}
	w = u = v=0 \quad \text{ at }\quad z=0,1.	
	\label{e:noslip}
\end{equation}

A well known result is that eqns. \eqref{e:rbc_eqn_nd} and \eqref{e:bc} are stable for sufficiently small $Ra.$ 
In particular, there exists a critical value $Ra_{c}$ which distinguishes whether an infintessimal perturbation will grow (unstable) or decay (stable). 
If $Ra<Ra_c,$ a conductive, static solution is stable. 
Conversely, convection occurs for $Ra>Ra_c$. $Ra_c$ is dependent on boundary conditions, problem geometry, and, crucially, $Ek$ and $Pr.$ 
For small $Ek$, the rotational constraint inhibits convection, and larger values of $Ra$ are needed to exceed $Ra_{c}$. 
The critical value $Ra_c$ increases with decreasing $Ek,$ following the scaling law $Ra_c\sim Ek^{-4/3}$.
	\begin{table}
		\begin{center}
		\begin{tabular}{|c|c|c|c|c|c|}
			\hline
			Body&$Ra$ &$Ek$ &$Pr$ &$Re$ &$Ro$\\
			\hline
			Earth's Core & $10^{24}$ &$10^{-15}$ & $0.1-0.01$ &  $10^{8}$ &$10^{-7}$\\
			\hline
			Full Sphere (Numerical) & $2.5\times10^{10}$ &$10^{-8}$ & $10^{-2}$&$5\times 10^{4}$ &$5\times10^{-4}$ \\
			\hline
			Plane Layer (Numerical) &$3\times 10^{13}$ &$5\times10^{-9}$&$1$&$6\times 10^{4}$ &$3\times10^{-4}$ \\
			\hline
			Cylinder (Water)&$4\times 10^{12}$ & $5\times 10^{-8}$ & $5.2$&$10^{4}$ &$5\times 10^{-4}$\\
			\hline

\end{tabular}
		\end{center}
		\caption[Example values for non-dimensional parameters]{Example values for the non-dimensional parameters. Values for Earth's core are estimates (eg. \citep{pO13,gW98}). Values for example studies in different geometries: a full sphere (numerical)\citep{cG19}, plane layer (numerical)\citep{jS24}, and cylinder (experiment)\citep{mM23}. 
		All are considered state of the art. 
		Due to the different geometries, the specific definitions of each parameter vary from study to study, although order of magnitude is unaffected.
}
\label{t:nondim_geo}
	\end{table}

When $Ra = Ra_c$, the flow will form a laminar, highly structured cellular pattern characterised by a single wavenumber $k_{c}$. 
In the limit of small $Ek,$ $k_c$ increases as $k_c\sim Ek^{-1/3}.$ The critical wavelength $\lambda_{c} = 2\pi/k_c$ decreases as $Ek^{1/3}$.
In general, the unstable modes oscillate on a fast timescale $Ek^{2/3}.$ If the most unstable mode is oscillatory (generally true for $Pr<0.68$), the associated frequency is $\omega_{c}.$
Table \ref{t:asymp} provides a summary of the behaviour of the critical parameters as $Ek\rightarrow 0$.

 \begin{table}
 \begin{center}
 \begin{tabular}{|c|c|c|c|c|}
 \hline
 $Ek$ Limit& $Ra_c$ & $k_{c}$ &$\lambda_{c}$ & $\omega_{c}$\\
 \hline
 $Ek\rightarrow \infty$ & $657.5$ & $2.22$ & 2.82 & 0 \\
 \hline
 $Ek\rightarrow 0$ & $8.70\times Ek^{-4/3}$ & $1.30\times Ek^{-1/3}$ & $4.82\times Ek^{1/3}$ & $Ek^{-2/3}$ \\
 \hline
 \end{tabular}
 \end{center}
 \caption[Asymptotic scaling of critical parameters with $Ek$]{Asymptotic scalings for the critical parameters in rotating convection at $Pr=1$. Note that at $Pr=1,$ the onset of convection is stationary such that $\omega_{c}=0$, however all timescales evolve on the $Ek^{2/3}$ scale. Values from \citep{sC61}.} 
 \label{t:asymp}
 \end{table}


In a numerical simulation, it is important to resolve $\lambda_{c},$ because it is the forcing scale.
It has been suggested that, as turbulence becomes more vigorous this scale may increase \citep{cG19,cG25}, although our findings in Chapter \ref{Oliver_2023} question whether this allows a relaxation of the resolution requirements \citep{tO23}.

As has been mentioned, geophysical parameters can be quite extreme; 
a value of $Ek = 10^{-15}$ indicates that $\lambda_{c}\sim 10^{-5}.$ Scaled to the core, this suggests a critical wavelength around $10$ m, far too small to resolve in any global-scale simulation. There are a multiple ways around this problem. Perhaps the most common, and the approach we employ in Chapter \ref{Oliver_2025}, is to perform sweeps of simulations at moderate $Ek$ (eg. $ 10^{-4}-10^{-6}$) and determine a scaling relationship with $Ek$. We can then extrapolate our results to $10^{-15}.$
Another approach is to leverage the smallness of $Ek$ and determine a hierarchy of balances in eqns. \ref{e:rbc_eqn_nd} that are asymptotically distinct in the limit $Ek\rightarrow 0.$ 
This approach has a successful history and was most notably used by Busse and collaborators in a series of papers \citep{fB70,fB86a,fB86c} in which they developed the cylindrical annulus model for spherical convection. 

For the plane layer problem at hand, a reduced set of equations (which we refer to as the ``reduced model'' or  ``reduced equations'') were derived by Julien et al. \citep{kJ98a}, with additional details provided in \citep{mS06} (see also \citep{kJ12b,sM21,aR14}).
They are valid in the asymptotic limit $Ek\rightarrow 0.$
The full model is presented and used in Chapter \ref{Oliver_2023} without derivation, so we will make a few remarks about the underlying physics here.

The asymptotic reduction is underpinned by a rescaling of length and timescales. The horizontal directions ($\hat{\mathbf{x}}$ and $\hat{\mathbf{y}}$) are rescaled to the length of the critical modes ($\lambda_{c}$),
and the time $t$ is rescaled to accommodate the fast timescales.
\begin{equation}
x,y \rightarrow Ek^{1/3}\; x^{*},y^{*}\qquad \qquad  z\rightarrow z^{*}\qquad\qquad t \rightarrow Ek^{2/3} \;t^{*}.
	\label{e:kj_rescale}
\end{equation}
The starred quantities are $O\lb 1\rb $ in the theory of Julien et al., and are the arguments for the dynamical fields. 
The fields themselves are also rescaled according to
\[\mathbf{u} \rightarrow Ek^{-1/3}\mathbf{u}^{*}\qquad\qquad \vartheta \rightarrow Ek^{1/3}\vartheta^{*},\]
where $\vartheta$ is a temperature fluctuation about the horizontal mean.
The main point we would like to highlight is that the reduced equations are informed by the Taylor-Proudman (TP) constraint. $Ek^{1/3}$ and $Ek^{2/3}$ are the lengths and times over which TP can be relaxed. 
The scalings in eqn. \eqref{e:kj_rescale} leverage this result to effectively shrink (spatially) and shorten (temporally) the solution domain.

The reduced model also introduces two new variables to the analysis. They are the vorticity and streamfunction. 
The vorticity is the curl of the velocity field,
\begin{equation}
	\pmb{\omega} = \nabla^{*} \times \mathbf{u}^{*}.
	\label{e:def_omega}
\end{equation}
In particular we require the vertical ($\zhat$) component,
\[\zeta = \pmb{\omega} \cdot \zhat.\]
The streamfunction $\psi$ is defined such that
\begin{equation}
	-\nabla^{*}\times \psi\zhat = \mathbf{u}_{\perp}^{*},
	\label{e:def_psi}
\end{equation} 
which is equivalent to the statements $\partial_{x^{*}} \psi = v^{*}$ and $\partial_{y^{*}}\psi=-u^{*}$.
As a result
\begin{equation}
	\nabla_{\perp}^{*2} \psi = \zeta,
	\label{e:nab_psi}
\end{equation}
 where $\nabla_{\perp}^{*2} = \lb \partial_{x^{*}}^{2} + \partial _{y^{*}}^{2}\rb $ is the horizontal laplacian.
 These definitions may seem to be an over-complication, but they are dynamically motivated. 
 Consider the dimensionless statement of geostrophy (eqn. \eqref{e:geos}).
 \[
	 \begin{aligned}
		&\frac{2}{Ek}\zhat \times \mathbf{u} \approx -\nabla p \implies\\
		&\frac{2}{Ek} \zhat \times \mathbf{u}^{*} \approx -\nabla^{*}p \implies\\
		&v^{*} \approx \frac{Ek}{2}\partial_{x}^{*}p \qquad \text{and} \qquad u \approx -\frac{Ek}{2}\partial_{y}^{*}p.
	 \end{aligned}
 \]
 By inspection, $\psi \approx \frac{1}{2}Ek\;p$. 
 We had previously stated that geostrophy has the property that flows run coincident to the isobars.
 Here is the formal statement: 
 At leading order, the streamfunction and pressure are equivalent (up to a factor of Ekman), which indicates that streamlines (the contours of constant streamfunction \citep{dT01}) are the contours of constant pressure. 
 By formulating the governing equations in terms of the streamfunction, we can exchange $u,v,$ and $p$ for the single variable $\psi.$ It is notationally convenient to express the reduced equations in terms of $\psi$ and $\zeta,$ although we note that $\zeta$ follows directly from $\psi$ via eqn. \eqref{e:nab_psi}.
 
 %The reduced set of equations are
%\begin{subequations}
%	\begin{equation}
%	\partial_{t}\zeta + J\ls\psi,\zeta\rs = \partial_{z}w + \nabla^{2}_{\perp}\zeta\\
%		\label{e:f3_1}
%	\end{equation}
%\begin{equation}
%	\partial_{t}w	 + J\ls\psi,w\rs = -\partial_{z} \psi + \frac{\Rat}{Pr} \theta + \nabla^{2}_{\perp}w
%	\label{e:f3_2}
%\end{equation}
%\begin{equation}
%	\partial_{t}\theta + J\ls\psi,\theta\rs+w\partial_{z}\overline{\Theta} = \frac{1}{Pr}\nabla_{\perp}^{2}\theta
%	\label{e:f3_3}
%\end{equation}
%\begin{equation}
%	\partial_{z}^{2}\overline{\Theta} = Pr\;\partial_{z}\lb\overline{w\theta}\rb
%	\label{e:f3_4}
%\end{equation}
%\begin{equation}
%	\nabla_{\perp} \cdot\mathbf{u}_{\perp} = 0
%	\label{e:f3_5}
%\end{equation}
%\begin{equation}
%	\nabla_{\perp}^{2}\psi = \zeta
%	\label{e:f3_6}
%\end{equation}
%	\label{e:f3}
%\end{subequations}

\subsection*{A comment on spherical geometries}
Plane layer geometries are exceedingly useful as \textit{local} models for geophysical systems. 
The above system, for example, is a useful representation for the polar region of a planet, where rotation and gravity are anti-aligned. 
However they can miss many important features such as boundary curvature or radial gravity.
The solution approach in a spherical geometry, such as a thick spherical shell relevant to the outer core, is very similar to the procedure above. Eqn. \eqref{e:rbc_1} has to be modified to account for a radial gravity, and the boundary conditions need to accommodate shell surfaces rather than planar walls, but the main features of the dynamical system remain. 
Crucially, the scaling laws for the critical parameters are unchanged, that is $Ra_{c}\sim Ek^{-4/3}$ and $k_{c}\sim Ek^{-1/3}$.
Rotation inhibits convection in the sphere just as it does in the plane layer. 
Unfortunately the asymptotic reduction approach of Julien et al. does not work for spherical geometries because the shell prescribes the horizontal dimensions. In Chapters \ref{Oliver_2025} and \ref{Oliver_2026} we solve the full equations (similar to eqns. \eqref{e:rbc_eqn_nd}) at finite values of $Ek$ and extrapolate to $Ek = 10^{-15}.$

\section{Consequences of Rotation and Geostrophy}
The previous section introduced geostrophy and the Taylor-Proudman constraint, and we discussed some of the dynamical consequences related to convection. 
In this section, we will discuss a few phenomenon particular to rotating flows which play important roles in the projects discussed in this dissertation.

\subsection*{Geostrophic Contours}
 \begin{figure}
	\begin{center}
		\includegraphics[width=0.55\textwidth]{Introduction/figures/Busse_Annulus}
	\end{center}
\caption{The Busse model for spherical convection. The sloped caps at the top and bottom approximate curvature on the spherical surface. Diagram is borrowed from \textit{Treatise on Geophysics}, 2nd Edition. Vol. 8 C.A. Jones, Copyright Elsevier 2015 \citep{cJ15}.}
	\label{f:b_an}
\end{figure}
In this section we discuss a well known phenomenon in rotating fluid mechanics related to the so-called ``geostrophic contours.''
The key result is that the geostrophic flow is constrained to lines of constant height, where height is the distance between bounding surfaces along the rotation axis. 
Here we provide a physical justification for this phenomenon, however a full mathematical treatment of the problem can be found in Greenspan \cite{hG68}.
We will motivate the discussion with the annulus geometry first studied by Busse \citep{fB70}, but the results are general. 

Figure \ref{f:b_an} displays the Busse annulus, which was first designed as a model for spherical convection. 
For the present discussion, our concern will be the sloped caps at the top and bottom, which approximate curvature along a spherical surface. 
Under the effects of rapid rotation, every fluid element contains a large amount of angular momentum.
In particular, consider a cylindrical filament of fluid--radius $r$-- that extends from the bottom cap to the top cap along the rotation axis.

The axial angular momentum per unit mass of the filament about its central axis is
\[L_{tot} = \alpha \omega r^{2} + \beta \Omega r^{2} = L_{\omega} + L_{\Omega}.\]
The first term is related to the motion of the element in the rotating frame, and is proportional to the vorticity $\omega$, as calculated in the rotating frame. 
The second term is due to the rigid body rotation and is proportional to $\Omega.$ 
$\alpha$ and $\beta$ are $O\lb 1\rb $ constants. The ratio $\omega/\Omega$ is approximately the Rossby number. The assumption for geostrophy is that $Ro\ll 1,$ so $L_{\omega}$ is much less than $L_{\Omega}.$ 
When the Ekman number is very small, the strength of viscosity is weak. Fluid elements exert only weak shear stresses on their neighbors, so $dL_{tot}/dt\approx 0$. Note that we have used the total derivative $d/dt,$ we are following a particular fluid element as it moves through the flow.

Now imagine that the filament were to move radially inwards. 
It must stretch to accommodate the greater distance between the caps. 
We assume incompressibility, so that as the filament stretches, it's radius must shrink.
Like the figure skater tucking their arms in, the filament's rotation rate must increase, however the effect is immense and ultimately untenable. 
$\Omega$ does not change, but due to the reduction in $r,$ $L_{\Omega}$ shrinks tremendously. 
To compensate $\omega$ must increase on the order of $1/Ro,$ which would immediately break the geostrophic constraint. 
Of course outward motion is similarly restricted. 

The effect in the annulus is that the geostrophic flow can only move azimuthally. 
More generally, geostrophic flow can only move along paths for which the total height -- the distance between the upper and lower surface -- is unchanged.
We refer to these paths as the geostrophic contours, and they are determined entirely by the geometry of the container and the axis of rotation.
In the annulus, as in a pure sphere or spherical shell, the geostrophic contours are circles centered about the origin. In a plane layer the contours are degenerate, and all horizontal motion is permitted.
In many geometries the contours do not form closed loops. They may terminate at a boundary or  end abruptly within the domain. 
Formally, such cases prohibit geostrophic motion. However turbulent flows, after temporal average, will still follow the contours. Dissipation and non-linearities near the terminations of the contours breaks the geostrophic constraint.

Chapters \ref{Oliver_2025} and \ref{Oliver_2026} consider spherical shells with boundary topography. 
Topography distorts the geostrophic contours. In our first study the topography was sufficiently small such that the deformations to the contours were slight. 
Although the effects are noticeable in the visualizations, there was little impact on the global flow. 
However, global scale topography can yield large deformations and the result can drastically change the heat, momentum, and angular momentum transport within the fluid. This is investigated in Chapter \ref{Oliver_2026}.
